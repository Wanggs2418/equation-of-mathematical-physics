\documentclass{article}
% \documentclass[twocolumn]{article}
\usepackage{ctex}
% \usepackage{cite}
\usepackage{geometry}  %页面布局
\usepackage{listings} %代码环境
\usepackage{color}
\usepackage[dvipsnames,svgnames,x11names]{xcolor} %整合12种色彩模式
\usepackage{setspace} %行距
\usepackage{marginnote} %边注
\usepackage{verbatim} %代码环境
\usepackage{paralist} %列表的行间距
\usepackage{amsmath} %公式,ams版本的equation环境可以嵌入次环境
\usepackage{amsthm} %定理,提供proof环境来输出证明
\usepackage{amsfonts} %\mathfrak和\mathbb需要
\usepackage{mathrsfs} %\mathscr{text}需要
\usepackage{graphicx} %插入图片的宏包
\usepackage{subfig} %图片共享有一个子标题
\usepackage{pstricks} %可以直接在文档中插入绘图命令
\usepackage{pspicture}
\usepackage{booktabs} %三线表横线的的粗细
\usepackage{tikz} %前端TikZ调用,底层PGF系统驱动
\usepackage{pst-pdf} %生成包含PSTricks图形的EPS,根据需要转为
\usepackage{multirow} %横跨几行的宏包
\usepackage{warpcol} %调整小数点和数位对齐工作
\usepackage{longtable} %表格太长,用longtable取代tabular
\usepackage{tabularx} %控制整个表格的宽度
\usepackage{rotating} %设置宽表格时使用,用sidewaystable环境替代table
% \usepackage[table]{xcolor}
\usepackage{colortbl} %表格颜色
\usepackage{xltxtra} %XeLaTex标志符号显示使用
\usepackage{texnames} %BibTeX标志使用
\usepackage{mflogo} %METAFONT宏包使用
\usepackage{natbib} %文献
\usepackage{hyperref} %超链接功能
\usepackage{enumitem}
\usepackage{fixdif} %2023年收录的包,重新定义数学模式下微分算符d,即\d
\usepackage[framed,numbered,autolinebreaks,useliterate]{mcode}

% 更改有序列表样式
\renewcommand{\labelenumi}{(\arabic{enumi})}
\setlist{nosep}

\linespread{1.5}
\geometry{a4paper,left=2.5cm, right=2.5cm, top=2.5cm, bottom=2.5cm}
\graphicspath{{E:/Latex/takehome/image}} %事先指定图片的路径

\abovedisplayskip=12pt plus 3pt minus 9pt
\abovedisplayshortskip=0pt plus 3pt
\belowdisplayskip=12pt plus 3pt minus 9pt
\belowdisplayshortskip=7pt plus 3pt minus 4pt
\captionsetup{font={small,stretch=1.25},justification=raggedright}

\begin{document}
\section{基本概念}
\subsection{概念}
偏微分方程(Partial Differential Equation)PDE

常微分方程(Ordinary Differential Equation)ODE

数理方程中主要研究二阶线性偏微分方程。
\subsection{偏微分方程的阶}
未知函数偏导数\textbf{最高阶}定义为方程的阶。

\subsection{偏微分方程的分类}
(1)齐次方程和非齐次方程
是否含有仅依赖于自变量的函数

(2)线性方程——方程中未知函数及其偏导数都是线性的,且未知函数系数依赖于自变量。

(3)拟线性方程——\textbf{最高阶偏导数}是线性的,但方程非线性。

(4)非线性方程——最高阶偏导是非线性的。

\subsection{习题}
(a)$(xy +5)u_{xyyz} + x^2yu_{xyz} + y^4u_{yyz} + x^2u + \ln xz = 0$

含$\ln xz$非齐次,最高阶4阶为$u_{xyyz}$,整体线性。

(b)$u_{xx}^2(u_{yy}+1)+2u_xu_y^3+\sin x = 0$

含$\sin x$非齐次,最高阶数2阶为$u_{xx},u_{yy},u_{xy}$,含有$u_{xx}^2(u_{yy}+1),u_xu_y^3$项,非线性。

(c)$u_{xyz}u_{xyyz}+6x^2y^2u_{xyz}+u_xy=0$

方程非线性,但是最高项4阶为$u_{xyz}u_{xyyz}$且是线性的,所以总体为拟线性。

注意:如果是$u_{xxyz}u_{xyyz}$最高阶则是非线性的。

\section{数学模型的建立及定解问题}
\subsection{三类典型方程}
\begin{enumerate}
    \item 波动方程:$u_{tt}=a^2(u_{xx})+u_{yy}$
    \item 热传导方程:$u_t=k(u_{xx}+u_{yy})$
    \item 拉普拉斯方程:$u_{xx}+u_{yy}=0$
\end{enumerate}

\section{两个自变量的二阶线性偏微分方程的分类与化简}
\subsection{按数学角度分类}

\begin{enumerate}
    \item 双曲型方程:$u_{tt}=a^2 u_{xx}$
    \item 抛物型方程:$u_t=ku_{xx}$
    \item 椭圆型方程:$u_{xx}+u_{yy}=0$
\end{enumerate}
\subsection{两个自变量得二阶线性PDE得化简}
对于二阶线性微分方程的一般形式为:
% \vspace*{-1em}
$$
    a(x,y)u_{xx}+b(x,y)u_{xy}+c(x,y)u_{yy}+d(x,y)u_x+e(x,y)u_y+f(x,y)u=g(x,y)
$$

% \vspace*{-1em}
考虑变量变换
$$
    \left\{
    \begin{aligned}
         & \xi = \xi(x,y)   \\
         & \eta = \eta(x,y)
    \end{aligned}
    \right.
$$

Jacobi行列式确保满足,$J \ne 0$,即为可逆变换:
% \vspace*{-1em}
$$
    A(\xi,\eta)u_{\xi\xi}+B(\xi,\eta)u_{\xi\eta}+C(\xi,\eta)u_{\eta\eta}+D(\xi,\eta)u_\xi+E(\xi,\eta)u_\eta+F(\xi,\eta)u=G(\xi,\eta)
$$

% \vspace*{-1em}
为了使得变换后的方程简化,则$\xi,\eta$应满足
$$
    \left\{
    \begin{aligned}
         & A = a\xi_x^2+b\xi_x\xi_y+c\xi_y^2=0     \\
         & C = a\eta_x^2+b\eta_x\eta_y+c\eta_y^2=0
    \end{aligned}
    \right.
$$

故只需$\xi,\eta$满足方程:
$$
    a\varphi_x^2+b\varphi_x\varphi_y+c\varphi_y^2=0
$$

对于
$$
    a\left(\frac{\varphi_x}{\varphi_y}\right)^2+b\frac{\varphi_x}{\varphi_y}+c =0
$$

与之对应的为如下方程,注意b后面的符号为"-"
% 数学粗体使用amsmath宏包支持的blodsymbol命令,boldmath和mathbf不太行
$$
    a\left(\frac{\d y}{\d x}\right)^2\boldsymbol{-}b\frac{\d y}{\d x}+c =0
    % a\left(\frac{\d y}{\d x}\right)^2-b\frac{\d y}{\d x}+c =0
$$
解出$\xi = \varphi_1(x,y),\eta= \varphi_2(x,y)$
最后代入下式转换后即得标准型方程
$$
    \left\{
    \begin{aligned}
         & u_x = u_{\xi}\xi_x +u_{\eta}\eta_x  \\
         & u_y = u_{\xi}\xi_y +u_{\eta}\eta_y  \\
         & u_{xx} = u_{\xi\xi}\xi_x^2 +2u_{\xi\eta}\xi_x\eta_x+u_{\eta\eta}\eta_x^2+u_{\xi}\xi_{xx}+u_{\eta}\eta_{xx} \\
         & u_{xy} = u_{\xi\xi}\xi_x\xi_y+u_{\xi\eta}(\xi_x\eta_y+\xi_y\eta_x)+u_{\eta\eta}\eta_x\eta_y+u_{\xi}\xi_{xy}+u_{\eta}\eta_{xy} \\
         & u_{yy} = u_{\xi\xi}\xi_y^2+2u_{\xi\eta}\xi_y\eta_y+u_{\eta\eta}\eta_y^2+u_{\xi}\xi_{yy}+u_{\eta}\eta_{yy}
    \end{aligned}
    \right.
$$


\subsection{习题}
1.判断类型并化为标准型

(a)$u_{xx}+2\cos xu_{xy} - \sin^2 xu_{yy} -\sin xu_y=0$

解 \quad 判断类型:$\Delta = b^2-4ac=(2\cos x)^2+4\sin^2 =4>0$,为双曲型方程

特征方程为
\begin{align*}
    \d y^2-2\cos x \d x\d y - \sin^2 x \d x^2=0 \\
    \left(\frac{\d y}{\d x}\right)^2-2\cos \left(\frac{\d y}{\d x}\right)-\sin^2x=0
\end{align*}

可解得
$$
    \left(\frac{\d y}{\d x}\right)_{1,2}=\cos x \pm 1
$$

积分可得$\sin x \pm x - y=c$,并令
$$
    \left\{
    \begin{aligned}
         & \xi = \sin x +x-y  \\
         & \eta = \sin x -x-y
    \end{aligned}
    \right.
$$

其中
$\xi_x = \cos x+1,\xi_y=-1,\xi_{xx}=-\sin x,\xi_{yy}=\xi_{xy}=0,\eta_x=\cos x-1,\eta_y=-1,\eta_{xx}=-\sin x,\eta_{yy}=\eta_{xy}=0$,代入可得
$$
    \left\{
    \begin{aligned}
         & u_y = u_{\xi}\xi_y +u_{\eta}\eta_y \\
         & u_{xx} = u_{\xi\xi}\xi_x^2 +2u_{\xi\eta}\xi_x\eta_x+u_{\eta\eta}\eta_x^2+u_{\xi}\xi_{xx}+u_{\eta}\eta_{xx} \\
         & u_{xy} = u_{\xi\xi}\xi_x\xi_y+u_{\xi\eta}(\xi_x\eta_y+\xi_y\eta_x)+u_{\eta\eta}\eta_x\eta_y+u_{\xi}\xi_{xy}+u_{\eta}\eta_{xy} \\
         & u_{yy} = u_{\xi\xi}\xi_y^2+2u_{\xi\eta}\xi_y\eta_y+u_{\eta\eta}\eta_y^2+u_{\xi}\xi_{yy}+u_{\eta}\eta_{yy}
    \end{aligned}
    \right.
$$

最后可得$u_{\xi\eta}=0$




\section{S-L问题}
\subsection{常见的通解形式}
求 $u''+\lambda u=0$得通解形式

(1)当 $\lambda <0$,方程通解是
$$
u(x)=Ae^{\sqrt{-\lambda }x}+Be^{-\sqrt{-\lambda }x}
$$

(2)当 $\lambda =0$,方程通解是
$$
u(x)=Ax + B
$$

(3)当 $\lambda >0$,方程通解是
$$
u(x)=A\cos \sqrt{\lambda }x+ B\sin \sqrt{\lambda }x
$$
\subsection{习题}
3.求下列S-L问题的本征值与本征函数
% \vspace*{-1em}
$$
    \left\{
    \begin{aligned}
         & x^2u''+3xu'+\lambda u=0,1<x<e \\
         & u(1)=0,u(e)=0
    \end{aligned}
    \right.
$$

解 \quad 方程各项乘以$x$,可得$S-L$方程:
$$
    \frac{d}{dx}(x^3 \frac{du}{dx}) + \lambda x u = 0
$$
此时,$p(x)=x^3,q(x)=0,s(x)=x.$

(1)当$\lambda=0$时,方程通解为
$$
    u(x) = \frac{C_1}{x^2} + C_2
$$
带入边界条件
$$
    \left\{
    \begin{aligned}
         & u(1)= C_1 + C_2 = 0           \\
         & u(e)= \frac{C_1}{e^2} + C_2=0
    \end{aligned}
    \right.
$$
方程组只有零解,即$C_1=C_2=0$,所以$u(x) \equiv 0$

(2)当$\lambda$ < 1且$\lambda \ne 0$,方程的特解形式为
$$
    u(x) = x^{\beta}
$$
带入方程中可得:
\begin{equation}
    \begin{aligned}
        \beta (\beta-1)x^{\beta} + 3 \beta x^{\beta} + \lambda x^{\beta} = 0 \\
        x^{\beta}[\beta (\beta-1) + 3 \beta + \lambda ] = 0                  \\
        \beta^2 + 2 \beta + \lambda = 0                                      \\
        \beta = \pm \sqrt{1-\lambda}
        \nonumber
    \end{aligned}
\end{equation}
方程通解为
$$  u(x)=C_1x^{\sqrt{1-\lambda}} + C_2x^{-\sqrt{1-\lambda}} $$
带入边界条件
$$
    \left\{
    \begin{aligned}
         & u(1)= C_1 + C_2 = 0                                        \\
         & u(e)= C_1e^{\sqrt{1-\lambda}} + C_2e^{-\sqrt{1-\lambda}}=0
    \end{aligned}
    \right.
$$
方程组只有零解,即$C_1=C_2=0$,所以$u(x) \equiv 0$

(3)当$\lambda$ > 1,由(2)可得,此时有
\begin{align}
    \beta^2 + 2 \beta + \lambda = 0 \nonumber \\
    \beta = \pm i\sqrt{1-\lambda}
    \nonumber
\end{align}
方程通解为
$$  u(x)=C_1x^{i\sqrt{1-\lambda}} + C_2x^{-i\sqrt{1-\lambda}} $$
注意到
$$  x^{\pm i\sqrt{1-\lambda} } = e^{\pm i\sqrt{1-\lambda} \ln x} = \cos(\sqrt{1-\lambda} \ln x) \pm i\sin(\sqrt{1-\lambda} \ln x) $$
方程通解为实部和虚部的线性组合
$$  u(x)=A \cos(\sqrt{1-\lambda} \ln x)  + B \sin(\sqrt{1-\lambda} \ln x) $$
带入边界条件,由u(1)=0可得$A=0$,由$u(e)=0$可得
$$
    B\sin(\sqrt{1-\lambda})  = 0
$$
从上式可得本征值
$$ \lambda_n =  1+ n^2 \pi^2,n=1,2,3 \dots$$
相应的本征函数为
$$ u_n(x) = \frac{1}{x} \sin(n \pi \ln x), n=1,2,3 \dots$$

\subsection{常数变易法}
对于$n$阶线性微分方程在已知其齐次方程通解的基础上,利用常数变易法求解。一般是将前$n-1$阶解的导数表达式中
所有含待定系数函数一阶导数项之和等于0。

对于n阶线性微分方程:
$$
    y^{(n)}(x)+p_1(x)y^{(n-1)}(x)+\cdots+p_{n-1}(x)y'+p_n(x)y=f(x)
$$

对应的齐次方程通解为:$y=C_1y_1+C_2y_2+\cdots+C_ny_n$

利用常数变易法设方程的解形式为:$y^{*}=C_1(x)y_1+C_2(x)y_2+\cdots+C_n(x)y_n$,其中$C_i(x)$为待定系数函数,对该解分别进行$1,2,\cdots,(n-1)$阶导,可得到$n-1$个方程。
$$
    \left\{
    \begin{aligned}
         & y=C_1y_1+C_2y_2+\cdots+C_ny_n                                        \\
         & y'=(C_1'y_1+C_2'y_2+\cdots+C_n'y_n) + (C_1y_1'+C_2y_2'+\cdots+C_ny_n')     \\
         & y''=(C_1''y_1+C_2''y_2+\cdots+C_n''y_n) + (C_1'y_1'+C_2'y_2'+\cdots+C_n'y_n') + (C_1y_1''+C_2y_2''+\cdots+C_ny_n'')       \\
         & \cdots   \\
         & y^{(n-1)}=(C_1^{(n-1)}y_1+C_2^{(n-1)}y_2+\cdots+C_n^{(n-1)}y_n) +\cdots+ (C_1y_1^{(n-1)}+C_2y_2^{(n-1)}+\cdots+C_ny_n^{(n-1)})
    \end{aligned}
    \right.
$$

分别令$n-1$个表达式中所有含待定函数一阶导数的项的和为0。
$$
    \left\{
    \begin{aligned}
         & C_1'(x)y_1(x)+C_2'(x)y_2(x)+\cdots+C_n'(x)y_n(x)  =0 \\
         & C_1'(x)y_1'(x)+C_2'(x)y_2'(x)+\cdots+C_n'(x)y_n'(x)  =0   \\
         & \cdots \\
         & C_1'(x)y_1^{(n-2)}(x)+C_2'(x)y_2^{(n-2)}(x)+\cdots+C_n'(x)y_n^{(n-2)}(x)  =0   \\
    \end{aligned}
    \right.
$$

其中最后一项,$n$阶导数为:
$$ 
y^{(n)}(x)+p_1(x)y^{(n-1)}(x)+\cdots+p_{n-1}(x)y'+p_n(x)y=C_1'(x)y_1^{(n-1)}(x)+C_2'(x)y_2^{(n-1)}(x)+\cdots+C_n'(x)y_n^{(n-1)}(x)  =f(x)
$$

$$
\begin{cases}
k_{11}x_1+k_{12}x_2+\cdots+k_{1n}x_n=b_1 \\
k_{21}x_1+k_{22}x_2+\cdots+k_{2n}x_n=b_2 \\
\cdots \\
k_{n1}x_1+k_{n2}x_2+\cdots+k_{nn}x_n=b_n \\
\end{cases}
$$

\section{积分变换}
求解无界区域或半无界区域上的问题,将$f(x)$经过某种的积分变换
$$
F(\lambda)=\int K(x,\lambda)f(x) \d x
$$

其中$\lambda$为参变量,$K(x,\lambda)$为积分变换核,$F(\lambda)$为$f(x)$的\textbf{像函数},$f(x)$为$F(\lambda)$的原像函数。
通过积分变换,将PDE转换为依赖参变量ODE的定解问题,求出ODE的解后,经过逆变换,得到PDE的解。

$f(x)$在$(-\infty,\infty)$内分段光滑且绝对可积。则积分 $\displaystyle F(\lambda)=\frac{1}{\sqrt{2\pi}} \int_{-\infty}^{+\infty}f(\xi)e^{i\lambda \xi } d \xi$,称为 $f(\xi)$的傅里叶积分变换,记作 $\tilde{F}[f(x)]$。积分 $\displaystyle f(x)=\frac{1}{\sqrt{2\pi}}\int_{-\infty}^{+\infty}F(\lambda)e^{-i\lambda x } d \lambda$称为 $F(\lambda)$的傅里叶积分逆变换,记作 $\tilde{F}^{-1}[F(\lambda)]$。

\vspace*{1em}
原理公式:$\displaystyle f(x)=\frac{1}{\sqrt{2\pi }}\int_{-\infty}^{+\infty}\left[\frac{1}{\sqrt{2\pi }}\int_{-\infty}^{+\infty}f(\xi )e^{i\lambda \xi }\d \xi \right]e^{-i\lambda x }\d \lambda $
\vspace*{1em}

\textbf{其中内层积分呢为正变换,外层积分为逆变换,最终变回自身}
\vspace*{0.5em}

误差函数说明:$\displaystyle  erf(z)=\frac{2}{\sqrt{\pi }}\int_0^z e^{-\eta ^2} \d \eta $,积分限为 $(0,z)$

余误差函数为:$\displaystyle  erfc(z)=\frac{2}{\sqrt{\pi }}\int_z^{+\infty} e^{-\eta ^2} \d \eta $,积分限为 $(z,+\infty)$

\vspace*{0.5em}
\textbf{误差函数+余误差函数=1},即 $erf(z)+erfc(z)=1$

\subsection{傅里叶变换的性质}
\begin{enumerate}
\item 线性变换:$\tilde{F}[\alpha f(x) + \beta g(x)]=\alpha \tilde{F}[f(x)]+\beta \tilde{F}[g(x)]$
\item 位移性质:$\tilde{F}[f(x-c)]=e^{i\lambda c}\tilde{F}[f(x)]$
\item 积分性质:$\displaystyle \tilde{F}[\int_{-\infty}^xf(\xi) \d \xi ]=-\frac{1}{i\lambda }\tilde{F}[f(x)]$
\item 微分性质:$\tilde{F}[f^{(n)}(x)]=(-i\lambda )^{n}\tilde{F}[f(x)]$
\item 卷积:$f(x),g(x)$都满足傅里叶变换条件,即 $F(\lambda)=\tilde{[f(x)]},G(\lambda )=\tilde{F}[g(x)]$,
\item 则积分 $\displaystyle \frac{1}{\sqrt{2\pi }}\int_{-\infty}^{+\infty}f(x-\xi )g(\xi )d \xi =\displaystyle \frac{1}{\sqrt{2\pi }}\int_{-\infty}^{+\infty}g(x-\xi )f(\xi )d \xi$
\end{enumerate}

\subsection{例题}
\subsubsection*{波动方程求解}
用傅里叶变换求解
$$
\begin{cases}
    u_{tt}=c^2u_{xx}, &-\infty <x <+\infty ,t>0\\
    u(x,0)=\varphi(x)  &\\
    u_t(x,0)=\psi(x) &\\
\end{cases}
$$

解: \quad 令 $\tilde{F}[u(x,t)]=V(\lambda ,t)$,对方程做傅里叶变换转换为:
$$
\begin{aligned}
    &\tilde{F}[u_{tt}] = c^2\tilde{F}[u_{xx}] \\
    &\frac{\d^2 V}{\d t^2 }+c^2(-i\lambda )^2V=   0 \\
    &V(\lambda ,t)=c_1(\lambda )\cos c\lambda t+c_2(\lambda )\sin c\lambda t
\end{aligned}
$$

将已知条件带入得,
$$
\begin{aligned}
    &\tilde{F}[u(x,0)]=\tilde{F}[\varphi (x)]=V(x,0)=F(\lambda ) \\
    &\tilde{F}[u_t (x,0)]\tilde{F}[\psi (x)]=V_t(x,0)=G(\lambda ) 
\end{aligned}
$$
$$
V(\lambda ,t)=F(\lambda )\cos c\lambda t+\frac{G(\lambda )}{c\lambda } \sin c\lambda t
$$

傅里叶逆变换,
$$
\begin{aligned}
    \cos c\lambda t = &\frac{e^{ic\lambda t}+e^{-ic\lambda t}}{2} \\
    \tilde{F}^{-1}[F(\lambda )\cos c\lambda t]=&\frac{1}{\sqrt{2\pi }}\int_{-\infty}^{+\infty}F(\lambda )\cos c\lambda t e^{-i\lambda x} \d \lambda \\
    =&\frac{1}{2\sqrt{2\pi }}\int_{-\infty}^{+\infty}F(\lambda )e^{ic\lambda t}e^{-i\lambda x} \d \lambda +\frac{1}{2\sqrt{2\pi }}\int_{-\infty}^{+\infty}F(\lambda )e^{-ic\lambda t}e^{-i\lambda x} \d \lambda \\
    =& \frac{1}{2}\tilde{F}^{-1}[F(\lambda )] e^{i\lambda ct}+\frac{1}{2}\tilde{F}^{-1}[F(\lambda )] e^{-i\lambda ct}
\end{aligned}
$$

根据位移公式 $\tilde{F}[f(x-c)]=e^{i\lambda ct}\tilde{F}[f(x)]$可得
$$
\begin{aligned}
    f(x-c)=&\tilde{F}^{-1}[\tilde{F}(f(x))]e^{i\lambda c} \\
    \varphi (x-ct)=&\tilde{F}^{-1}[\tilde{F}(\varphi (x))] e^{i\lambda ct}\\
    =& \tilde{F}^{-1}[F(\lambda )] e^{i\lambda ct}\\
\end{aligned}
$$

带入上式可得
$$
\begin{aligned}
   \textrm{原式} =& \frac{1}{2}\tilde{F}^{-1}[F(\lambda )] e^{i\lambda ct}+\frac{1}{2}\tilde{F}^{-1}[F(\lambda )] e^{-i\lambda ct} \\
    =& \frac{1}{2}\varphi (x-ct) + \frac{1}{2}\varphi (x+ct)
\end{aligned}
$$

\subsection{拉普拉斯变换(L-T)}
傅里叶变换得局限性:它要求被变换函数定义在$(-\infty,+\infty)$上,其次要求被变换函数在两端要趋于零,且趋向速度比较快,即满足绝对可积的条件。

而一些涉及到时间的问题,只需研究$t>0$的变化,不能直接用傅里叶变换。此时若满足 $|f(t)| \le Me^{s_0t}$(具有指数阶),且分段光滑条件不变,则可利用拉普拉斯变换。
$$
f_1(t)= 
\begin{cases}
    f(t)e^{-st},& t>0,s>s_0>0 \\
    0,& t \le 0 \\
\end{cases}
$$

此时作傅里叶变换,得拉普拉斯正变换:

$$
\begin{aligned}
\tilde{F}[f(t)e^{-st}]=&\frac{1}{\sqrt{2\pi }}\int_{0}^{+\infty}f(t)e^{st}e^{i\lambda t}\d t=\frac{1}{\sqrt{2\pi }}\int_{0}^{+\infty}f(t)e^{-(s-i\lambda )t} \d t \\
\overset{p=(s-i\lambda )}{=} & \frac{1}{\sqrt{2\pi }}\int_{0}^{+\infty}f(t)e^{-pt}\d t=F(p)
\end{aligned}
$$

逆变换,一般不用,常借助性质求解。

$$
\begin{aligned}
f(t)=\frac{1}{2\pi i}\int_{s-i\infty}^{s+i\infty}F(p)e^{pt}\d p \\
\end{aligned}
$$

原理公式:

$$
\begin{aligned}
f(t)=\frac{1}{2\pi i}\int_{s-i\infty}^{s+i\infty}\left[ \int_{0}^{+\infty}f(\eta) e^{-p\eta }\d \eta \right]e^{pt}\d p \\
\end{aligned}
$$

\subsection{拉普拉斯变换性质}
\begin{enumerate}
    \item 线性变换:$\tilde{L}[\alpha f(t)+\beta g(t)]=\alpha \tilde{L}[f(t)]+\beta \tilde{L}[g(t)]$
    \item 微分性质1:$\displaystyle \tilde{L}[f^{n}(t)]=p^n\{ \tilde{L}[f(t)]-\frac{f(0^+)}{p}-\cdots - \frac{f^{(n-1)}(0^+)}{p^n} \}$
    \\ 微分性质2:$\displaystyle \frac{\d^n F(p)}{\d p^n}=\tilde{L}[(-t)^{n}f(t)],\tilde{L}[f(t)]=\int_{0}^{+\infty}f(\eta) e^{-p\eta }\d \eta=F(p)$
    \item 积分性质1:$\displaystyle \varphi =\int_0^tf(\tau )\d \tau,\tilde{L}[\varphi (t)]=\frac{1}{p}\tilde{L}[f(t)]$
    , $\tilde{L}[\varphi '(t)]=\tilde{L}[f(t)]=p\tilde{L}[\varphi (t)]$
    \\ 积分性质2:$\displaystyle F(p)=\tilde{L}[f(t)],\int_p^{+\infty}F(s)=\tilde{L}[\frac{1}{t}f(t)]$
    \item 位移性质:$\displaystyle F(p-p_0)=\tilde{L}[e^{p_0tf(t)}]$
    \item 延迟性质(重点):$\tilde{L}[f(t-c)]=e^{-pc}\tilde{L}[f(t)]=e^{-pc}F(p),c>0$
    \item 相似性质:$\displaystyle \tilde{L}[f(at)]=\frac{1}{a}F(\frac{p}{a}),a>0$
    \item 卷积性质:将$\displaystyle \int_0^tf(t-\xi )g(\xi ) \d \xi =\int_0^tg(t-\xi )f(\xi ) \d \xi$,记为$f*g(t)=g*f(t)$。
\end{enumerate}

\textbf{拉普拉斯变换是线性变换},其中积分微分$1,2$性质主要在于\textbf{是先进行积分或微分操作,还是先进行拉普拉斯变换。}

卷积性质说明:卷积的拉普拉斯变换等于拉普拉斯变换后的乘积;乘积的变换等于变换后的卷积。

\subsection{常见的拉普拉斯变换}
具体如下:
\vspace*{-1em}
%使用amsmath包
\begin{flalign}     
    &(1) f(t) =1 ,F(p)=\tilde{L}[1]=\frac{1}{p^2}&
    \nonumber \\
    &(2) f(t) = t^n, F(p) =\tilde{L}[t^n]=\frac{n!}{p^{n+1}}& 
    \nonumber \\
    &(3) f(t) = \cos wt,F(p)=\tilde{L}[\cos wt] = \frac{p}{p^2+w^2}& 
    \nonumber \\
    &(4) f(t) = \sin wt,F(p)=\tilde{L}[\sin wt] = \frac{w}{p^2+w^2}&
    \nonumber 
\end{flalign}

\section{格林函数法}
\subsection{格林函数法解非线性ODE边值问题}
前面借助常数变异法解非齐次ODE初值问题


\end{document}

